\documentclass{article}
\usepackage[utf8]{inputenc}
\usepackage[dvipsnames]{xcolor}
\usepackage[a4paper, total={6in, 10in}]{geometry}

\title{Functors Applicatives}
\author{dannymaate}
\date{July 2022}

\begin{document}

\maketitle

\section{Functors}
Functors increase the level of generality in Haskell. Think of them as functions over a range of parameterised types such as lists, trees. \newline

\noindent Functors have both an \textcolor{Mulberry}{infix notation} where \texttt{\textcolor{Emerald}{(<\$>) = fmap}, e.g. fmap (>1) [1,2,3]} and a \textcolor{Mulberry}{prefix notation} where \texttt{\textcolor{Emerald}{fmap (g.h) = fmap g . fmap h}, e.g. (>1) <\$> (Just 1)}

\paragraph{Functor Laws}
\begin{enumerate}
  \item \texttt{fmap id = id}
  \item \texttt{fmap (g.h) = fmap g . fmap h}
\end{enumerate}

\paragraph{Application of Composition}
\texttt{(f.g) x, f.g \$ x, f \$ g x, f \$ g \$ x}

\paragraph{Structural Induction}
Suppose S is some \textcolor{Mulberry}{recursively defined structure} (e.g. [a] or Tree a) that has \textcolor{Mulberry}{substructure} (e.g. sublist, subtree) and there is \textcolor{Mulberry}{partial order}{e.g. length or number of nodes}. 
\\
\textcolor{Mulberry}{Structural induction} implies if...
\begin{enumerate}
  \item P is true for \textcolor{Mulberry}{all minimum structures}, and     (Base Case)
  \item P(x') is true for x' any \textcolor{Mulberry}{immediate substructure} of x (Induction)
\end{enumerate}

\section{Applicatives}
Functors can map a function over each element in a structure. Suppose we wish to generalise the idea to allow functions with \textcolor{Mulberry}{any number of arguments} to be mapped. 

\paragraph{Applicative Laws}
\begin{enumerate}
  \item \texttt{pure id <*> x = x} \textcolor{Mulberry}{\emph{mapping identity has no effect}}
  \item \texttt{pure (g x) = pure g <*> pure x} \textcolor{Mulberry}{\emph{pure distribution}}
  \item \texttt{x <*> pure y = pure (\textbackslash{}g -\> g y) <*> x} \textcolor{Mulberry}{\emph{effectful function disregards evaluation order}}
  \item \texttt{x <*> (y <*> z) = (pure (.) <*> x <*> y) <*> z} \textcolor{Mulberry}{\emph{associativity}}
\end{enumerate}

\paragraph{Corollary}
\verb|g <$> x = pure g <*> x|

\paragraph{Effectful Programming}
Applicative functors abstract the idea of applying \textcolor{Mulberry}{pure functions} to \textcolor{Mulberry}{effectful arguments}, where the effects depend on the underlying functor. Effects may be: possibility of failure, having many ways to succeed or I/O actions. \newline 
 
\subsubsection*{Applicative Style}
Note that: \textcolor{Emerald}{\texttt{fmap g x = pure g <*> x}}, so this means 
\begin{quote}
\textcolor{Emerald}{\texttt{pure g <*> x1 <*> x2 <*> ... <*> xn}}\par 
\end{quote}
is same as
\begin{quote}
\newline\textcolor{Emerald}{\texttt{g <\$> x1 <*> x2 <*> ... <*> xn}}
\end{quote}
\end{document}
