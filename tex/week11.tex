\documentclass{article}
\usepackage[utf8]{inputenc}
\usepackage[dvipsnames]{xcolor}
\usepackage[a4paper, total={6in, 10in}]{geometry}

\title{Interactive Programming}
\author{dannymaate}
\date{13 July 2022}

\begin{document}
\maketitle

\section{IO}
Interactive programs by nature require the side effects of taking additional inputs and producing additional outputs. IO is \textcolor{Rhodamine}{\emph{impure}} (writing to files, throwing exceptions). 

\subsubsection*{Purity in Haskell}
In Haskell, interactive programs are pure functions that accept the current \emph{state of the world} as an argument and produce a modified world as the result, as well as a result value. 

\begin{verbatim}
    type IO a = World -> (a, World)
\end{verbatim}
\emph{a state transformer with a state world.}
Expressions with type IO are \textcolor{Rhodamine}{\emph{actions}}

\subsubsection*{Basic IO Actions}
\texttt{getChar :: IO Char} \newline
\texttt{getChar = ...} \emph{\textcolor{Brown}{reads a character}} \newline

\noindent \texttt{putChar :: Char -> IO()} \newline
\texttt{putChar c = ...} \emph{\textcolor{Brown}{writes a character c.}}
\textcolor{Rhodamine}{\emph{Note: () is an IO dummy value that doesn't return anything}} \newline

\noindent \texttt{getLine :: IO string} \newline 
\noindent \texttt{putStr :: String -> IO ()} \newline
\noindent \texttt{putStrLn :: String -> IO ()}

\paragraph{Monad}
Say you didn't care for the return value and want to perform a \textcolor{Rhodamine}{\emph{bind}}.\textcolor{Emerald}{\emph{You'd use the (>>)}}.

\subsubsection*{Useful String Functions}
\begin{verbatim}
    show :: Show a => a -> String
    read :: Read a => String -> a
    
    > show 123 
    "123"
    
    > read "123" :: Int 
    123

    
    
\end{verbatim}
\end{document}